\subsection{Why does proposition 1 hold?}\label{ssec:prop1}

\begin{enumerate}[(a)]
\item Why does proposition 1 hold? How can a solution be extracted from $\rho$? Give a comprehensible explanation. (A formal proof is not necessary.)
\end{enumerate}
Let's recall proposition 1 from~\cite{MF19}:

\paragraph{Proposition 1}
\emph{Let $\rho$ be an interval valuation.}
\emph{Then, if all simple constraints of a CSP $\mathcal{P}$ are satisfied by $\rho$ then there exists a (point) solution of $\mathcal{P}$ and, hence, $\mathcal{P}$ is satisfiable.}

%\paragraph{}
%Or informally: If we find a $\rho$ which satisfies all constraints $c \in C$, there exists a solution for the corresponding CSP.

\paragraph{}
An interval valuation is a mapping from a variable $x_i \in X$ to a non-empty interval $\rho(x_i)$ with $p(x_i) \subseteq D_i \subset \mathbb{Z}$.
Per definition 1, a solution $A$ to a CSP $\mathcal{P}$ is an n-tuple $A = \langle a_1, ..., a_n \rangle$ with (1) $a_i \in D_i$, s.t. (2) all constraints $c \in C$ are satisfied.~\cite{MF19}

Let $\rho$ be an interval valuation of a CSP $\mathcal{P}$, which satisfies all constraints $c \in C$.
Using the above definition of solutions of a CSP and interval valuations, we can see that $\forall a_i \in A: a_i \in \rho(x_i) \subseteq D_i$, hence we can pick a single value $a_i$ from each $\rho(x_i)$ to extract a solution $A$.
We know $A$ is then a solution, because both properties of a solution hold:

\begin{enumerate}[(1)]
    \item $\forall a_i \in A: a_i \in D_i$, and
    \item $A$ satisfies all constraints $c \in C$, because $\rho$ already does.
\end{enumerate}
