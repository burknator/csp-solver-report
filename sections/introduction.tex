\section{Introduction}

\emph{Constraint satisfaction problems} model problems in many areas in science and industry.
A \emph{CSP} is modelled with a set of variables, each one having it's own domain, as well as constraints which describe the relationships between the variables.~\cite{MF19}

The domains are not necessarily bounded or unbounded, finite or infinite.
But in this task, only simplified CSPs are considered, they are bounded as well as finite.~\cite{MF19}
Because the task only considers simple CSPs, we use the terms ``CSP'' and ``simple CSP'' interchangeably.

This report uses the same definitions as the task description, which can be found in the appendix~\ref{sec:apx:definitions} for easier retrieval.

\subsection{Task Description}

The task was to write a tool which gets as input a single CSP in form of a plaintext file.
After receiving the input, the program is supposed to find out of there exists a solution for the given CSP and if so state on possible solution.

Two algorithms $\mathcal{A}$ and $\mathcal{B}$ were given in the task description, which are capable of determining if a CSP is satisfiable and give a solution if so.

We won't reiterate on the algorithms directly here, since they can be found in the task description.
However, we will briefly describe our implementation of these algorithms in section~\ref{?}.
