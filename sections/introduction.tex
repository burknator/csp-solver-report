\section{Introduction}

\emph{Constraint satisfaction problems} model problems in many areas in science and industry.
A \emph{CSP} is modelled with a set of variables, each one having it's own domain, as well as constraints which describe the relationships between the variables.~\cite{MF19}

The domains are not necessarily bounded or unbounded, finite or infinite.
But in this task, only simplified CSPs are considered, which are bounded as well as finite.~\cite{MF19}

Because the task only considers simple CSPs, we use the terms ``CSP'' and ``simple CSP'' interchangeably.

Also note that this report uses the same definitions as the task description.

\subsection{Task Description \& Overview}

Six exercises are supposed to be solved, two of which are optional.
In these exercises we were tasked to show why interval satisfaction shows CSP satisfaction, that algorithm~$\mathcal{A}$ is sound and complete, that the given type of simple CSPs are NP-complete and examine whether the algorithm can handle simple CSPs with infinite, real-valued bounds.
Our solutions to those can be found in section~\ref{sec:exercise-solutions}.

Additionally, we were tasked with writing a tool which gets as input a single CSP in form of a plaintext file.
After receiving the input, the program is supposed to find out if the given CSP is satisfiable, and if so, provide a solution.
Two algorithms $\mathcal{A}$ and $\mathcal{B}$, which are capable of performing the above task, were given as part of the task description.
The section~\ref{ssec:algorithms} describes our implementation of these algorithms, section~\ref{ssec:parser-impl} describes the implementation of the parser.

Lastly, you will find our conclusion in section~\ref{sec:conclusion}.
