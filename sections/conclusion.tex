\section{Conclusion}\label{sec:conclusion}

We showed that interval satisfaction is a sufficient criterion to tell if a simple CSP is satisfiable.
By displaying the search space as a tree structure, and mapping the algorithm's steps to movements in this tree, we showed that its sound and complete.

Further, we were able to show that the given type of CSPs are NP-complete by reducing them to boolean SAT problems.

We concluded that the algorithm~$\mathcal{A}$ works correctly on real-valued, infinite bounds, as long as it terminates, which is not guaranteed with these types of bounds.

The algorithms~$\mathcal{A}$ and $\mathcal{B}$ were implemented and described, including a parser to read the input CSP. Due to lack of time, we weren't able to work on many of the optional tasks, like enhancing the algorithm by implementing lacy clause evaluation, but we tried to sketch some ideas on lazy clause evaluation and decision heuristics.
