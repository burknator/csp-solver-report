\subsection{Soundness and Completeness of $\mathcal{A}$}
\begin{enumerate}[(b)]
\item Is algorithm $\mathcal{A}$ sound and complete, i.e. does $\mathcal{A}$ stop on each simple CSP with the correct result? Give reasons for your answer.
\end{enumerate}

Algorithm $\mathcal{P}$ is both sound and complete, because it traverses the search space in a complete manner and only returns correct results.
To explain this, we will use a tree structure representing the search space and argue that the algorithm traverses this tree in a way that allows it to always stop with the correct result.

\begin{figure}[H]
    \centering
    \begin{tikzpicture}[initial text=, ->,>=stealth',shorten >=0pt,auto,node distance=2cm, semithick, scale = 0.9, transform shape]
\begin{scope}
    \node [state] (p0) {$\rho^0$};
    \node [state] (p1) [below of=p0, left=2] {$\rho^1$};
    \node [state] (p2) [below of=p0, right=2, dashed]{$\rho^2$};
    \node [state, fill=red!20] (p3) [below of=p1, left=2] {$\rho^3$};
    \node [state] (p4) [below of=p1, right=2, dashed] {$\rho^4$};
    \node [state] (p5) [below of=p4, left=1, dotted] {$\rho^5$};
    \node [state] (p6) [below of=p4, right=1, dotted] {$\rho^6$};
    \node [state] (p7) [below of=p3, left=1, dotted] {$\rho^7$};
    \node [state] (p8) [below of=p3, right=1, dotted] {$\rho^8$};
    
    \path (p0) edge [left] node {$x_i$} (p1)
          (p0) edge [right, dashed] node {$x_i$} (p2)
          (p1) edge [left] node {$x_{ii}$} (p3)
          (p1) edge [right, dashed] node {$x_{ii}$} (p4)
          (p4) edge [left, dotted] node {$x_{i}$} (p5)
          (p4) edge [right, dotted] node {$x_{i}$} (p6)
          (p4) edge [bend left=20, dashed] node {$b_1$} (p2)
          (p3) edge [bend left=20, dashed] node {$b_2$} (p4)
          (p3) edge [left, dotted] node {$x_i$} (p7)
          (p3) edge [right, dotted] node {$x_i$} (p8);

\end{scope}
\end{tikzpicture}

    \caption{
        A figure showing the search space as a tree-structure.
        With $\rho^i$ being interval valuations, $x, y \in X$ and $b_i$ being backtracking steps.
    }
    \label{fig:search-tree}
\end{figure}


\paragraph{Structure}
Fig.~\ref{fig:search-tree} shows the search space as a tree.
Each node represents an interval valuation, with $\rho^0$ being the initial interval valuation.
Non-bend edges running from a top node to a bottom node represent a parent-child relationship, bent edges show a backtrack alternative.
The children of a node are two interval valuations created from their parent as part of step 3 of algorithm $\mathcal{A}$.
The variable at the edges is the variable whose interval was split.
Dashed (\dashed) elements are known by the algorithm already, but not used, while dotted (\dotted) elements are not yet known at all, though they exist theoretically.
Red nodes ($\rho^{3,7,8}$) represent interval valuations under which at least one constraint is false, while green nodes ($\rho^{4,5,6}$) represent valuations which satisfy all constraints.

$\rho^3$ is the current interval valuation, $\rho^4$ and $\rho^2$ are currently on the backtrack-alternative stack, in that order.
$\rho^3$ is also a dead-end in the search space, meaning that at least one constraint is false under it.
The bend edge from $\rho^3$ to $\rho^4$ shows that $\rho^4$ is the next available backtrack option, and the one from $\rho^4$ to $\rho^2$ shows that $\rho^2$ is the option after that.

The algorithm dictates that $\rho^{i+1} \cup \rho^{i+2} = \rho^i$ and $\rho^{i+1} \cap \rho^{i+2} = \emptyset$, meaning that the two children of a parent are disjoint and together form the whole parent.
Which is important, because otherwise parts of the variable intervals would be missing from, or occur more than once in the search space.

\paragraph{Traversal}
The algorithm traverses the tree by asserting the interval valuations represented as nodes in the tree.
Step 3 of the algorithm would first create two children of a node and then, by asserting one of those, travers the edge from the parent node to the child node which represents the asserted interval valuation.
Backtracking as described in step 2 of the algorithm is a lateral traversal along a bent edge in the tree.
Step 1 of the algorithm does not result in any traversal inside the tree, since it does not change the current interval valuation.
For the same reason no traversal happens when the algorithm stops with some result.

\paragraph{Solution in the Tree}
As described in~\ref{ssec:prop1}, if a CSP is satisfiable, its solutions are part of the interval valuations.
Which in our case means that the solutions are part of the tree.

\paragraph{}
The next lemma will help argue that it's okay for the algorithm to stop exploring a branch when the root-valuation falsifies at least one constraint.

\begin{lemma}\label{lemma:false-branches}
    Once an interval valuation $\rho$ is found which falsifies a single constraint, the entire branch rooting in that interval valuation can be ignored by the search for a result.
    It is impossible for the truth value to change by further exploring this branch, because the condition $\max \rho(x) < \min \rho(y) + k$ for each false simple bound stays true (see def.~4~\cite{MF19}). 
\end{lemma}

\paragraph{Proof}

Let $\rho$ be in interval valuation under which at least one constraint is false, and have false simple bounds have the form $x \geq y + k$.

Were the algorithm $\mathcal{A}$ to execute the decision step when a constraint is false under the current interval valuation, only the following cases could occur:

Let $x \in X$ be a variable that is part of at least one of the false simple bounds.
Let $\rho'(x)$ and $\rho''(x)$ be interval valuations created from $\rho(x)$ according to step 3 of algorithm $\mathcal{A}$.
We know that $\max \rho'(x), \max \rho''(x) \leq \max \rho(x)$ holds, because the algorithm dictates that $\rho'(x) \cup \rho''(x) = \rho(x)$.
Thus, $\max \rho(x) < \min \rho(y) + k$ still holds for all simple bounds that $x$ is part of, which means their truth value stays unchanged.

Let $y \in X$, $\rho(y)$, $\rho'(y)$ and $\rho''(y)$ be created analogous to before.
We know that $\min \rho'(y)$, $\min \rho''(y) \geq \min \rho(y)$ holds, because the algorithm dictates that $\rho'(y) \cup \rho''(y) = \rho(y)$.
Thus, $\max \rho(x) < \min \rho(y) + k$ still holds for all simple bounds that $y$ is part of, which means their truth value stays unchanged.

If the interval valuation of a variable which is not part of any of the false simple bounds is split, the truth values of these simple bounds are obviously not going to change. $k$ cannot change. $\square$


\paragraph{}
The algorithm traverses the search tree by asserting the interval valuations which are represented as nodes in the tree.
Whenever such an interval valuation is asserted, all possible cases are handled by the algorithm:

\begin{itemize}
    \item
        \textbf{All constraints are true.}
        In this case, the algorithm stops correctly with \emph{CSP $\mathcal{P}$ is satisfiable} because it found either a point solution or a set of solutions.
    
    \item
        \textbf{At least one constraint is false.}
        Because of Lemma~\ref{lemma:false-branches} we know that this branch of the search space is a dead-end.
        In this case, the algorithm backtracks to a previously created but unused interval valuation to explore the branch of the search space that roots there.
        If there's no such backtrack alternative, it means there are no branches left to explore, thus the algorithm stops correctly with \emph{CSP $\mathcal{P}$ is unsatisfiable}.
    \item
        \textbf{Neither of the two.}
        In this case there's no way to tell if the CSP is satisfiable or not.
        The algorithm opens up two new branches in the search tree in step 3, stores one as a backtrack alternative and continues the search in the other one.
\end{itemize}

\paragraph{Summary}
The algorithm stops exploring branches of the tree which are known to not contain a solution, butexplores different branches (via backtracking) until an interval valuation is found that satisfies all constraints.
Once that happens, it stops with \emph{CSP $\mathcal{P}$ is satisfiable}, which is correct as described in~\ref{ssec:prop1}.
It's also correct that the algorithm stops exploring this branch because, using that valuation, all solutions of that branch can be created.
The algorithm will find a solution if one exists, because it keeps searching in parts of the tree that can contain them.

If the algorithm runs out of branches to explore before that happens, it stops with \emph{CSP $\mathcal{P}$ is unsatisfiable}, which is also correct because we know that dead-end branches cannot contain solutions.

%\paragraph{}
%\todo{Das stimmt dochx, oder?}
%As shown in section~\ref{ssec:prop1}, if a CSP $\mathcal{P}$ is satisfiable, each $\rho$ under which none of the constraints $c \in C$ are false contains a solution.
%The fact that the algorithm ignores parts of the search space which are known to not contain a solution, but keeps searching in parts that still could, it will get to a result any time.
%
%\todo[inline]{Completeness and soundness noch mal extra herausstellen, also welche Teil jetzt fuer was sorgt}


%Der Algorithmus durchsucht alle Äste des Suchbaums die eine Lösung enthalten können, bis eine Lösung gefunden wurde.
%Weil der Suchbaum alle Lösungen enthält, ist der Algorithmus vollständig.
%Und weil der Algorithmus nur mit satisfiable anhält, wenn alle Constraints wahr sind, ist er auch sound.


%Each $c \in C$ can be in one of three separate states:
%
%\begin{enumerate}
%    \item \texttt{true}
%    \item \texttt{false}
%    \item \texttt{inconclusive}
%\end{enumerate}
%
%From this we can conclude, that, at all times, these seven possibilities of combinations of truth values of all $c \in C$ can occur:
%
%\begin{enumerate}
%    \item All true
%    \item All false
%    \item All inconclusive
%    \item Some true, some false
%    \item Some true, some inconclusive
%    \item Some false, some inconclusive
%    \item Some true, some false, some inconclusive
%\end{enumerate}
%
%The only times the truth values of the constraints can change at all are when the interval valuation changes, this can happen in steps 2 and 3.
%Whenever this happens, the algorithm immediately goes to step 1.
%
%Step 1 correctly stops with $\mathcal{P}$ \emph{is satisfiable} when all $c \in C$ are true.
%
%The algorithm stops either in step 1 or step 2.
%
%
%Steps 1 and 3 behave the same way, no matter how many steps the algorithm has made up to the current point of execution (there is no ``state'' except for $\rho$).

%\subsubsection{Soundness}
%
%By contradiction.
%
%Let's assume a satisfiable CSP $\mathcal{P}$.
%Let's further assume that the Algorithm $\mathcal{A}$ applied to that CSP stops with the answer \emph{$\mathcal{P}$ is unsatisfiable}.
%
%For this to happen, the algorithm must have come across the step 2 and discover that there are no previously unused decision steps, because this is the only way the algorithm stops with ``unsatisfiable''.
%
%This can only happen when the cases 2, 4, 6 or 7 occur, and step 3 was executed as many times as step 2.
%
%
%\subsubsection{sldkf;dkfhlskjh}
%
%Im ersten Schritt diese Faelle:
%
%1. Alle True ->
%stoppt korrekt mit satisfiable
%
%2. Alle oder wenigstens einer False -> 
%Schritt 2
%
%3. Einige oder keine true, einige oder alle inkonklusiv ->
%Schritt 3
%
%
%Im zweiten Schritt (backtracking):
%
%1. Decision Level = 0 -> stoppt, mit unsatisfiable (z.b. der Fall direkt bei der 1. Iteration wenigstens ein constraint false ist)
%
%2. Decision Level > 0 -> 
%Verwende neues $\rho$, dann Schritt 1
%
%
%Im dritten Schritt:
%
%1. Es muss hier immer eine Variable geben, dessen $\rho$ wir aufteilen können.
%Das ist auch der Fall, denn wenn in einem simple bound $x \geq y + k$ $|\rho(x)| = |\rho(y)| = 1$ gilt, ist dieses entweder true oder false, und somit nicht inkonklusiv. Wenn dies für alle Variablen gilt, sind auch alle SImple bounds entweder true oder false, was bedeuten würde das wir gar nicht in diesem Schritt landen würden.
